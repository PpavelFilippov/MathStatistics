\section{Постановка задачи}
\textbf{Даны 5 распределений:}
\begin{itemize}
\item Нормальное распределение: \begin{gather*}N(x,0,1)\end{gather*}
\item Распрделение Коши: \begin{gather*}C(x,0,1)\end{gather*}
\item Распределение Стьюдента:\begin{gather*}t(x,0,3)\end{gather*}
\item Распределение Пуассона: \begin{gather*}P(k,10)\end{gather*}
\item Нормальное распрделение: \begin{gather*}U(x,-\sqrt{3}, \sqrt{3})\end{gather*}
\end{itemize}
\begin{enumerate}
    \item Сгенерировать выборки размером 20 и 100 элементов.
Построить для них боксплот Тьюки. Для каждого распределения определить долю выбросов экспериментально (сгенерировав выборку, соответствующую распределению 1000 раз, и вычислив среднюю долю выбросов) и сравнить с результатами, полученными теоретически.
\end{enumerate}


\section{Теоретическая информация}


\subsection{Распределения}
\begin{itemize}
    \item  Нормальное распределение \begin{gather*}
 N(x,0,1) = \frac{1}{\sqrt{2\pi}}e^{-\frac{x^{2}}{2}}
\end{gather*} 
    \item  Распределение Коши  
    \begin{gather*}
    C(x,0,1) = \frac{1}{\pi}\frac{1}{x^{2} + 1}
\end{gather*} 
    \item  Распределение Стьюдента   \begin{gather*}
 t(x,0,3) = \frac{Y_0}{\sqrt{\displaystyle\sum_{i=0}^{3} Y_i^{2}}}, Y_i \sim N(0,1)
\end{gather*} 
    \item  Распределение Пуассона
    \begin{gather*}
 P(k,10) = \frac{10^{k}}{k!}e^{-10}
\end{gather*} 
    \item  Нормальное распределение
    \begin{gather*}
U(x,-\sqrt{3}, \sqrt{3}) = \begin{cases}
\frac{1}{2\sqrt{3}}, |x| \leq \sqrt{3}\\
0,  |x| > \sqrt{3} \\ 
\end{cases}
\end{gather*}

\end{itemize}


\subsection{Боксплот Тьюки}
\subsubsection{Определение}
Боксплот (англ. box plot) — график, использующийся в описательной статистике, компактно изображающий одномерное распределение вероятностей.
\subsubsection{Описание}
Такой вид диаграммы в удобной форме показывает медиану, нижний и верхний квартили и выбросы.
Несколько таких ящиков можно нарисовать бок о бок, чтобы визуально сравнивать одно распределение с другим; их можно располагать как горизонтально, так и вертикально. Расстояния между различными частями ящика позволяют определить степень разброса (дисперсии) и асимметрии данных и выявить выбросы
\subsubsection{Построение} Границами ящика служат первый и третий квартили, линия в середине ящика — медиана. Концы усов —
края статистически значимой выборки (без выбросов). Длину «усов» определяют разность первого квартиля и полутора межквартильных расстояний и сумма третьего квартиля и полутора межквартильных расстояний. Формула имеет вид \begin{gather*}
    X_1 = Q_1 - \frac{}{}(Q_3 - Q_1) ,  X_2 = Q_3 - \frac{}{}(Q_3 - Q_1) \\
\end{gather*},где \textit{X\textsubscript{1}} - нижняя граница уса, \textit{X\textsubscript{2}} - верхняя граница уса, \textit{Q\textsubscript{1}} - первый квартиль, \textit{Q\textsubscript{3}} - третий квартиль. Данные, выходящие за границы усов (выбросы), отображаются на графике в виде маленьких кружков

\section{Изображения}


    \begin{figure}
        \centering
        \includegraphics{fig/T_N.png}
        \caption{Нормальное распределение}
        \label{fig:enter-label}
    \end{figure}

    \begin{figure}
        \centering
        \includegraphics{fig/C_N.png}
        \caption{Распределение Коши}
        \label{fig:enter-label}
    \end{figure}

    \begin{figure}
        \centering
        \includegraphics{fig/T_t.png}
        \caption{Распределение Стьюдента}
        \label{fig:enter-label}
    \end{figure}

    \begin{figure}
        \centering
        \includegraphics{fig/T_P.png}
        \caption{Распределение Пуассона}
        \label{fig:enter-label}
    \end{figure}

    \begin{figure}
        \centering
        \includegraphics{fig/T_U.png}
        \caption{Равномерное распределение}
        \label{fig:enter-label}
    \end{figure}