\usepackage{amsmath}
\section{Постановка задачи}
\textbf{Даны 5 распределений:}
\begin{itemize}
\item Нормальное распределение: \textit{N(x,0,1)}
\item Распрделение Коши: \textit{C(x,0,1)}
\item Распределение Стьюдента:\textit{ t}
\item Распределение Пуассона: \textit{P(k,10)}
\item Нормальное распрделение: \textit{U(x,-\sqrt{3}, \sqrt{3})}
\end{itemize}
\begin{enumerate}
    \item Необходимо сгенерировать выборки размером 10,50 и 1000 элементов. Построить на одном рисунке гистограмму и график плостности распределения.
    \item Сгенерировать выборки размером 10, 100 и 1000 элементов. Для каждой вычислить следующие характеристики положения данных: \bar{x}, \textit{med}x, z\textsubscript{R}, z\textsubscript{Q}, z\textsubscript{tr}. Повторить такие вычисления 1000 раз для каждой выборки и найти среднее характеристик положения и их квадратов:
    \begin{gather*}
 E(z) = \bar{z}
\end{gather*} 
    Вычислить оценку дисперсии по формуле:
    \begin{gather*}
    D(z) = \bar{z^{2}} - \bar{z}^{2}
\end{gather*} 
    Представить полученные данные в виде таблицы
\end{enumerate}


\section{Теоретическая информация}


\subsection{Распределения}
\begin{itemize}
    \item  Нормальное распределение \begin{gather*}
 N(x,0,1) = \frac{1}{\sqrt{2\pi}}e^{-\frac{x^{2}}{2}}
\end{gather*} 
    \item  Распределение Коши  
    \begin{gather*}
    C(x,0,1) = \frac{1}{\pi}\frac{1}{x^{2} + 1}
\end{gather*} 
    \item  Распределение Стьюдента 
    \item  Распределение Пуассона
    \begin{gather*}
 P(k,10) = \frac{10^{k}}{k!}e^{-10}
\end{gather*} 
    \item  Нормальное распределение
    \begin{gather*}
U(x,-\sqrt{3}, \sqrt{3}) = \begin{cases}
\frac{1}{2\sqrt{3}}, |x| \leq \sqrt{3}\\
0,  |x| > \sqrt{3} \\ 
\end{cases}
\end{gather*}

\end{itemize}


\subsubsection{Определение}
Гистогра́мма в математической статистике — это один из графических методов исследования рядов распределения значений случайной величины.
\subsection{Характеристики положения}
\begin{itemize}
    \item Выборочное среднее    \begin{gather*}
    \bar{x} = \frac{1}{n}\displaystyle\sum_{i=1}^{n} x_i
\end{gather*} 
    \item Выборочная медиана
     \begin{gather*}
med x = \begin{cases}
x\textsubscript{(\textit{l}+1)}, n = 2l + 1\\
\frac{x\textsubscript{(\textit{l})} + x\textsubscript{(\textit{l}+1)}}{2},  n = 2l \\ 
\end{cases}
\end{gather*}
    \item Полусумма экстремальных выборочных элементов \begin{gather*}
        z\textsubscript{R} = \frac{x\textsubscript{(1)} + x\textsubscript{(\textit{n})}}{2}
    \end{gather*}
    \item Полусумма квартилей 
    Выборочная квартиль z\textsubscript{p} порядка \textit{p} определяется формулой 
    \begin{gather*}
        z\textsubscript{p} = \begin{cases}
x\textsubscript{([\textit{np}]+1)}, np дробное\\
x\textsubscript{(\textit{np})},  np целое \\ 
\end{cases}
    \end{gather*}

    полусумма квартилей \begin{gather*}
        z\textsubscript{Q} = \frac{z\textsubscript{\frac{1}{4}} + z\textsubscript{\sfrac{3}{4}}}{2}
    \end{gather*}
    
    \item Усеченное среднее \begin{gather*}
            z\textsubscript{tr} = \frac{1}{n-2r}\displaystyle\sum_{i=r+1}^{n-r} x_(i), r \approx \frac{n}{4}
    \end{gather*}

\end{itemize}

\subsubsection{Характеристики рассеяния}
Выборочная дисперсия \begin{gather*}
    D = \frac{1}{n}\displaystyle\sum_{i=1}^{n}(x_i - \bar{x})^{2}
\end{gather*}

\section{Репозиторий}

\section{Результаты}
\subsection{Гистограмма и график плотности распределения}
\subsection{Характеристики положения и рассеяния}
\begin{table}[H]
    \centering
   
    \begin{tabular}{|l|l|l|l|l|l|}
    \hline
         &  x &   \textit{med}x   &   z\textsubscript{R}  &   z\textsubscript{Q}  &   z\textsubscript{tr}\\ \hline
         n = 10& & & & & \\ \hline
         \textit{E(z)} & -0.003426 & -0.009699 & 0.003866 & 0.310776 & 0.267591 \\ \hline
         \textit{D(z)} & 0.101463       &       0.137621       &       0.182218       &        0.125949       &        0.117248 \\ \hline
        \textit{E(z)} + \sqrt{D(z)} &[-0.321959, 0.315107] & [-0.380672, 0.361274] & [-0.423004, 0.430736] & [-0.044117, 0.665669] & [-0.074824, 0.610006] \\ \hline
         n = 100& & & & & \\ \hline
         \textit{E(z)} & & & & & \\ \hline
         \textit{D(z)} & & & & & \\ \hline
         \textit{E(z)} + \sqrt{D(z)} & & & & & \\ \hline
         n = 1000& & & & & \\ \hline
         \textit{E(z)} & & & & & \\ \hline
         \textit{D(z)} & & & & & \\ \hline
         \textit{E(z)} + \sqrt{D(z)} & & & & & \\ \hline
    \end{tabular}
     \caption{Таблица характеристик для нормального распределения}
    \label{tab:my_label}
\end{table}
\begin{table}[H]
    \centering
   
    \begin{tabular}{|l|l|l|l|l|l|}
    \hline
         &  x &   \textit{med}x   &   z\textsubscript{R}  &   z\textsubscript{Q}  &   z\textsubscript{tr}\\ \hline
         n = 10& & & & & \\ \hline
         \textit{E(z)} & & & & & \\ \hline
         \textit{D(z)} & & & & & \\ \hline
        \textit{E(z)} + \sqrt{D(z)} & & & & & \\ \hline
         n = 100& & & & & \\ \hline
         \textit{E(z)} & & & & & \\ \hline
         \textit{D(z)} & & & & & \\ \hline
         \textit{E(z)} + \sqrt{D(z)} & & & & & \\ \hline
         n = 1000& & & & & \\ \hline
         \textit{E(z)} & & & & & \\ \hline
         \textit{D(z)} & & & & & \\ \hline
         \textit{E(z)} + \sqrt{D(z)} & & & & & \\ \hline
    \end{tabular}
     \caption{Таблица характеристик для распределения Коши}
    \label{tab:my_label}
\end{table}
\begin{table}[H]
    \centering
    \begin{tabular}{|l|l|l|l|l|l|}
    \hline
         &  x &   \textit{med}x   &   z\textsubscript{R}  &   z\textsubscript{Q}  &   z\textsubscript{tr}\\ \hline
         n = 10& & & & & \\ \hline
         \textit{E(z)} & & & & & \\ \hline
         \textit{D(z)} & & & & & \\ \hline
        \textit{E(z)} + \sqrt{D(z)} & & & & & \\ \hline
         n = 100& & & & & \\ \hline
         \textit{E(z)} & & & & & \\ \hline
         \textit{D(z)} & & & & & \\ \hline
         \textit{E(z)} + \sqrt{D(z)} & & & & & \\ \hline
         n = 1000& & & & & \\ \hline
         \textit{E(z)} & & & & & \\ \hline
         \textit{D(z)} & & & & & \\ \hline
         \textit{E(z)} + \sqrt{D(z)} & & & & & \\ \hline
    \end{tabular}
     \caption{Таблица характеристик для распределения Стьюдента}
    \label{tab:my_label}
\end{table}
\begin{table}[H]
    \centering
   
    \begin{tabular}{|l|l|l|l|l|l|}
    \hline
         &  x &   \textit{med}x   &   z\textsubscript{R}  &   z\textsubscript{Q}  &   z\textsubscript{tr}\\ \hline
         n = 10& & & & & \\ \hline
         \textit{E(z)} & & & & & \\ \hline
         \textit{D(z)} & & & & & \\ \hline
        \textit{E(z)} + \sqrt{D(z)} & & & & & \\ \hline
         n = 100& & & & & \\ \hline
         \textit{E(z)} & & & & & \\ \hline
         \textit{D(z)} & & & & & \\ \hline
         \textit{E(z)} + \sqrt{D(z)} & & & & & \\ \hline
         n = 1000& & & & & \\ \hline
         \textit{E(z)} & & & & & \\ \hline
         \textit{D(z)} & & & & & \\ \hline
         \textit{E(z)} + \sqrt{D(z)} & & & & & \\ \hline
    \end{tabular}
     \caption{Таблица характеристик для распределения Пуассона}
    \label{tab:my_label}
\end{table}
\begin{table}[H]
    \centering
    \begin{tabular}{|l|l|l|l|l|l|}
    \hline
         &  x &   \textit{med}x   &   z\textsubscript{R}  &   z\textsubscript{Q}  &   z\textsubscript{tr}\\ \hline
         n = 10& & & & & \\ \hline
         \textit{E(z)} & & & & & \\ \hline
         \textit{D(z)} & & & & & \\ \hline
        \textit{E(z)} + \sqrt{D(z)} & & & & & \\ \hline
         n = 100& & & & & \\ \hline
         \textit{E(z)} & & & & & \\ \hline
         \textit{D(z)} & & & & & \\ \hline
         \textit{E(z)} + \sqrt{D(z)} & & & & & \\ \hline
         n = 1000& & & & & \\ \hline
         \textit{E(z)} & & & & & \\ \hline
         \textit{D(z)} & & & & & \\ \hline
         \textit{E(z)} + \sqrt{D(z)} & & & & & \\ \hline
    \end{tabular}
     \caption{Таблица характеристик для равномерного распренделения }
    \label{tab:my_label}
\end{table}